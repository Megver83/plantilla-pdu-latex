% Comando para línea punteada
\newcommand{\sectionline}{%
 \vspace{-8pt}
  \tikz{\draw[azulUCN, line width=0.8pt, dash pattern=on 3pt off 2pt] (0,0)--(\linewidth,0); }%
  \vspace{8pt}
}

% =====================================
% CONFIGURACIÓN DE SIUNITX
% =====================================
\sisetup{round-mode = places,
         round-precision = 2, % Redondea a dos decimales
         exponent-mode = fixed,
         fixed-exponent = 0,
         zero-decimal-to-integer, % Elimina ceros decimales en enteros
         output-decimal-marker = {,}, % separador decimal como coma
         detect-all} % hereda la tipografía del entorno

% =====================================
% BIBLIOTECAS DE TIKZ
% =====================================
\usetikzlibrary{shadows}
\usetikzlibrary{arrows.meta, decorations.pathreplacing}

% =====================================
% FRANJA AZUL - BORDE IZQUIEDO
% =====================================
\AtBeginShipout{%
  \AtBeginShipoutUpperLeft{%
    \begin{tikzpicture}[remember picture, overlay]
      % Franja vertical izquierda
      \fill[azulUCN] (current page.north west) rectangle ([xshift=0.5cm]current page.south west);

      % Logo en esquina superior derecha (opcional)
      \node[anchor=north east] at (current page.north east)
      {\includegraphics[width=2.3cm]{images/logo-vicerrectoria.png}};
    \end{tikzpicture}%
  }%
}

% =====================================
% COMANDOS PARA LA SECCIÓN RESULTADOS
% =====================================

% ---- Definición de tipos de columnas para simplificar ----
\newcolumntype{J}[1]{>{\justifying\setlength{\parindent}{0pt}\arraybackslash}m{#1}}
\newcolumntype{C}[1]{>{\centering\arraybackslash}m{#1}}

% ---- Comando para generar fila de indicador ----
\newcommand{\FilaIndicador}[5]{%
  #1 & #2 &
  \MostrarValor{#3}{#5} &
  \MostrarValor{#4}{#5}%
  \\ \hline
}

% ---- Comando para generar fila de subindicador ----
\newcommand{\FilaSubIndicador}[5]{%
  \rowcolor{black!7}
  \multicolumn{1}{!{\color{gray!40}\vrule width 10pt}>{\hspace{0.01mm}}l!{\color{black}\vrule width 0.8pt}}{#1} &
  #2 &
  \MostrarValor{#3}{#5} &
  \MostrarValor{#4}{#5}%
  \\ \arrayrulecolor{black}\hline
}

% ---- Cabecera estándar de las tablas ----
\newcommand{\CabeceraTabla}{%
  \rowcolor{azulUCN}
  \textcolor{white}{\textbf{Indicador}} &
  \textcolor{white}{\textbf{Descripción}} &
  \textcolor{white}{\textbf{Medición}} &
  \textcolor{white}{\textbf{Meta}} \\
  \hline
}

% ---- Entorno para tabla de objetivos ----
\newenvironment{tablaObjetivos}{%
  \begin{center}
  \footnotesize
  \begin{tabular}{|J{6cm}|J{6cm}|C{1.8cm}|C{1cm}|}
  \hline
  \CabeceraTabla
}{%
  \end{tabular}
  \end{center}
}

% ---- Comando para Objetivo Específico ----
\newcommand{\ObjEsp}[3]{%
  \subsubsection*{%
    \textcolor{azulUCN}{\textbf{•  Objetivo Específico #1:}} #2
    \addcontentsline{toc}{subsubsection}{Objetivo Específico #1}
  }%
  \vspace{-4mm} % reduce el espacio entre el título y la tabla
  \begin{tablaObjetivos}
    #3
  \end{tablaObjetivos}
}

% =====================================
% OBJETIVO ESTRATÉGICO 
% =====================================
\newmdenv[
  linecolor=naranjaUCN,
  linewidth=2pt,
  topline=false,
  bottomline=false,
  rightline=false,
  leftline=true,
  innertopmargin=5pt,
  innerbottommargin=5pt,
  innerleftmargin=8pt,
  innerrightmargin=5pt,
  skipabove=10pt,
  skipbelow=10pt,
  backgroundcolor=white
]{objetivoBox}

% =====================================
% COMANDO PARA ÁREA DE DESARROLLO
% =====================================
\newcommand{\AreaDesarrollo}[2]{%
  \vspace{6mm}
  {\LARGE \textcolor{rojoUCN}{\textbf{#1}}}%
  \vspace{1mm}
  \addcontentsline{toc}{subsection}{#1} 
  #2
}

% ---- Macro unificada para objetivo estratégico (ahora con conclusión) ----
\newcommand{\ObjetivoEstrategico}[5]{%
  \begin{objetivoBox}
    \vspace{2mm}
    \ifstrempty{#3}{%
      {%
        \color{naranjaUCN}\textbf{Objetivo Estratégico #1:}\\[1mm]
        \addcontentsline{toc}{subsection}{Objetivo Estratégico #1}
        \color{naranjaUCN}\normalfont #2
      }%
    }{%
      \begin{minipage}[t]{0.8\textwidth}
        {\color{naranjaUCN}\textbf{Objetivo Estratégico #1:}}\\[1mm]
        \addcontentsline{toc}{subsection}{Objetivo Estratégico #1}
        {\color{naranjaUCN}\normalfont #2}
      \end{minipage}%
      \hfill
      \begin{minipage}[t]{0.22\textwidth}
        \vspace{0pt}
        \begin{center}
          \graficoCumplimientoChico{#3}%
        \end{center}
      \end{minipage}
    }%
    \vspace{-6mm}
    #4
    \ifstrempty{#5}{}{%
      \vspace{-2mm}
      \begin{tcolorbox}[colback=blue!7, colframe=white,
                        title=Conclusión, fonttitle=\bfseries,
                        boxrule=0pt,
                        sharp corners,
                        width=\textwidth]
        {\color{black!70} \textbf{Análisis}:#5}
        \addcontentsline{toc}{subsubsection}{Análisis}
      \end{tcolorbox}
    }
  \end{objetivoBox}
}

% ============================
% GRÁFICOS DE CUMPLIMIENTO
% ============================
\newcommand{\graficoCumplimientoChico}[1]{%
\begin{tikzpicture}
  \def\radio{1.1cm}
  \pgfmathparse{
    ifthenelse(#1<0.6,"BAJO",
      ifthenelse(#1<0.8,"MEDIO",
        ifthenelse(#1<0.95,"SATISFACTORIO","SOBRESALIENTE")
      )
    )
  }
  \let\categoria\pgfmathresult
  \pgfmathparse{
    ifthenelse(#1<0.6,"red!60!black",
      ifthenelse(#1<0.8,"yellow!80!black",
        ifthenelse(#1<0.95,"green!50!black","blue!60!black")
      )
    )
  }
  \let\colorCategoria\pgfmathresult
  \draw[gray!20, line width=6pt] (0,0) circle (\radio);
  \draw[\colorCategoria, line width=6pt] (90:\radio) arc (90:{90-(#1*360)}:\radio);
  \node at (0,0) [align=center] {
    \scriptsize Cumplimiento\\
    \SI{\fpeval{#1*100}}{\percent}
  };
  \node[below=0.4em] at (0,-\radio) {\scriptsize \textbf{\categoria}};
\end{tikzpicture}%
}

% ======================
% MISIÓN Y VISIÓN
% ======================
\newcommand{\MisionVision}[2]{%
  \begin{center}
    \begin{minipage}[t]{0.48\textwidth}
      \centering
      \textbf{MISIÓN}
      \begin{justify}
        #1
      \end{justify}
    \end{minipage}\hfill
    \begin{minipage}[t]{0.48\textwidth}
      \centering
      \textbf{VISIÓN}
      \begin{justify}
        #2
      \end{justify}
    \end{minipage}
  \end{center}
}

% =========================================
% GRÁFICO Y TABLA RESUMEN INDICADORES (PÁG RESUMEN)
% =========================================
\newcommand{\GraficoCumplimiento}[5]{%
\begin{center}
\begin{tikzpicture}[scale=0.85]
  \pgfmathsetmacro{\vA}{min(100,#1*100)}
  \pgfmathsetmacro{\vB}{min(100,#2*100)}
  \pgfmathsetmacro{\vC}{min(100,#3*100)}
  \pgfmathsetmacro{\vD}{min(100,#4*100)}
  \pgfmathsetmacro{\vE}{min(100,#5*100)}
  \begin{axis}[
    ybar,
    width=16cm,
    height=4cm,
    bar width=45pt,
    title={\textbf{\% Cumplimiento por Área de Desarrollo}},
    title style={yshift=12pt, font=\small},
    symbolic x coords={
      {Identidad y\\Vida Universitaria},
      {Formación a lo\\Largo de la Vida},
      {Investigación,\\Tecnología e\\Innovación},
      {Vinculación con\\el Medio},
      {Dirección\\Estratégica}
    },
    xtick=data,
    ymin=0,
    ymax=100,
    axis y line=none,
    axis x line=bottom,
    axis line style={-},
    enlarge x limits=0.1,
    grid=none,
    nodes near coords={
        \pgfmathprintnumber[fixed,precision=2]{\pgfplotspointmeta}\%
    },
    every node near coord/.append style={
        font=\footnotesize,
        above,
        yshift=2pt
    },
    x tick label style={
        align=center,
        font=\footnotesize
    },
    axis on top
  ]
    \addplot[fill=naranjaUCN!80] coordinates {
      ({Identidad y\\Vida Universitaria},\vA)
      ({Formación a lo\\Largo de la Vida},\vB)
      ({Investigación,\\Tecnología e\\Innovación},\vC)
      ({Vinculación con\\el Medio},\vD)
      ({Dirección\\Estratégica},\vE)
    };
  \end{axis}
\end{tikzpicture}
\end{center}
}

% ---- Comando para mostrar valor o porcentaje ----
\newcommand{\MostrarValor}[2]{%
  \ifstrempty{#1}{%
    N/A%
  }{%
    \ifstrequal{#2}{porcentaje}{%
      \SI{\fpeval{#1*100}}{\percent}%
    }{%
      \num{#1}%
    }%
  }%
}

% ---- Cabecera de tabla ----
\newcommand{\CabeceraIndicadores}{%
  \rowcolor{azulUCN}
  \textcolor{white}{\textbf{Indicador}} &
  \textcolor{white}{\textbf{Medición}} &
  \textcolor{white}{\textbf{Meta}} &
  \textcolor{white}{\textbf{Cumplimiento}} &
  \textcolor{white}{\textbf{Delta}} \\
  \hline
}

% ---- Entorno tabla ----
\newenvironment{TablaIndicadores}{%
  \begin{center}
  \footnotesize
  \renewcommand{\arraystretch}{1.2}
  \setlength{\tabcolsep}{3pt}
  \begin{tabular}{|p{8cm}|c|c|c|c|}
  \hline
  \CabeceraIndicadores
}{%
  \end{tabular}
  \end{center}
}

% ---- Comando para una fila ----
\newcommand{\FilaTablaIndicadores}[4]{%
  #1 &
  \MostrarValor{#2}{#4} &
  \MostrarValor{#3}{#4} &
  \IfStrEq{#2}{}{}{%
    \IfStrEq{#3}{}{}{%
      \pgfmathsetmacro{\cumraw}{#2/#3}%
      \pgfmathsetmacro{\cum}{min(\cumraw,1)}%
      \ifdim \cum pt<0.6pt
        \cellcolor{rojoPastel}\MostrarValor{\cum}{porcentaje}%
      \else
        \ifdim \cum pt<0.8pt
          \cellcolor{amarilloPastel}\MostrarValor{\cum}{porcentaje}%
        \else
          \ifdim \cum pt<0.95pt
            \cellcolor{verdePastel}\MostrarValor{\cum}{porcentaje}%
          \else
            \cellcolor{azulPastel}\MostrarValor{\cum}{porcentaje}%
          \fi
        \fi
      \fi
    }%
  } &
  \IfStrEq{#2}{}{}{%
    \IfStrEq{#3}{}{}{%
      \pgfmathsetmacro{\delta}{#2-#3}%
      \MostrarValor{\delta}{#4}%
    }%
  }%
  \\
  \hline
}

% ---- Comando para el área ----
\newcommand{\AreaIndicadores}[2]{%
  \rowcolor{naranjaUCN!25}
  \multicolumn{5}{|c|}{\textbf{Área de Desarrollo: #1}} \\
  \hline
  #2
}

% -------------------------
% TABLA ANEXO
% -------------------------
\newcommand{\CabeceraTablaAnexo}{%
  \rowcolor{azulUCN}
  \textcolor{white}{\textbf{Nombre del indicador}} &
  \textcolor{white}{\textbf{Forma de cálculo}} &
  \textcolor{white}{\textbf{Ponderación}} &
  \textcolor{white}{\textbf{Fuente de información}} &
  \textcolor{white}{\textbf{Frecuencia de actualización}} \\
  \hline
}

\newcommand{\FilaTablaAnexo}[5]{%
  #1 & #2 & \SI{\fpeval{#3*100}}{\percent} & #4 & \capitalisewords{#5} \\ \hline
}

\newcommand{\ObjEstrategicoAnexo}[2]{%
  \rowcolor{naranjaUCN!25}
  \multicolumn{5}{|p{\dimexpr5\linewidth/5-2\tabcolsep}|}{%
    \textbf{Objetivo Estratégico #1: #2}%
  } \\
  \hline
}

\newcommand{\TablaAnexo}[1]{%
  \begin{center}
  \scriptsize
  \renewcommand{\arraystretch}{1.2}
  \setlength{\tabcolsep}{3pt}
  \begin{tabular}{|C{4cm}|C{6cm}|c|C{2cm}|C{2.25cm}|}
  \hline
  \CabeceraTablaAnexo
  #1
  \end{tabular}
  \end{center}
}

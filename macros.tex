\setlength{\parindent}{0pt} % Quita sangrías de todos los párrafos
\setlength{\parskip}{0.5em} % Opcional: agrega espacio entre párrafos



% =====================================
% FRANJA AZUL - BORDE IZQUIEDO
% =====================================
\definecolor{UoDBlue}{RGB}{0,46,92}
\AtBeginShipout{%
  \AtBeginShipoutUpperLeft{%
    \begin{tikzpicture}[remember picture, overlay]
      % Franja vertical izquierda
      \fill[UoDBlue] (current page.north west) rectangle ([xshift=0.5cm]current page.south west);
      
      % Logo en esquina superior derecha (opcional)
      \node[anchor=north east] at (current page.north east)
      {\includegraphics[width=2.5cm]{images/logo vicerrectoria.png}};
    \end{tikzpicture}%
  }%
}


% =====================================
% COMANDOS PARA LA SECCIÓN RESULTADOS
% =====================================
\newcommand{\ObjEspProcess}[1]{#1}

% ---- Definición de tipos de columnas para simplificar ----
\newcolumntype{J}[1]{>{\justifying\setlength{\parindent}{0pt}\arraybackslash}m{#1}}
\newcolumntype{C}[1]{>{\centering\arraybackslash}m{#1}}

% ---- Comando para generar fila de indicador ----
\newcommand{\FilaIndicador}[5]{%
  #1 & #2 &%
  \ifstrempty{#3}{N/A}{
    \ifstrequal{#5}{porcentaje}{\SI{\fpeval{#3*100}}{\percent}}{\num{#3}}%
  }%
  &
  \ifstrempty{#4}{N/A}{
    \ifstrequal{#5}{porcentaje}{\SI{\fpeval{#4*100}}{\percent}}{\num{#4}}%
  }%
  \\ \hline
}
% \newcommand{\FilaIndicador}[5]{%
%   #1 & #2 & #3 & #4 \\ \hline
% }

\newcommand{\FilaSubIndicador}[5]{%
  \rowcolor{black!7}
  % Primera columna con borde izquierdo grueso y borde derecho fino (entre primera y segunda columna)
  \multicolumn{1}{!{\color{gray!40}\vrule width 10pt}>{\hspace{0.01mm}\color{black!60}}l!{\color{black!90}\vrule width 0.5pt}}{#1}
  % Segunda columna
  & \color{black!60}#2
  % Tercera columna
  &
  {
    \color{black!60}
    \ifstrempty{#3}{N/A}{
      \ifstrequal{#5}{porcentaje}{\SI{\fpeval{#3*100}}{\percent}}{\num{#3}}%
    }
  }
  % Cuarta columna
  &
  {
    \color{black!60}
    \ifstrempty{#4}{N/A}{
      \ifstrequal{#5}{porcentaje}{\SI{\fpeval{#4*100}}{\percent}}{\num{#4}}%
    }
  }
  \\ \hline
}

% ---- Cabecera estándar de las tablas ----
\newcommand{\CabeceraTabla}{%
  \rowcolor{UoDBlue}
  \textcolor{white}{\textbf{Indicador}} & 
  \textcolor{white}{\textbf{Descripción}} & 
  \textcolor{white}{\textbf{Medición}} & 
  \textcolor{white}{\textbf{Meta}} \\
  \hline
}

% ---- Entorno para tabla de objetivos con letra ligeramente más pequeña ----
\newenvironment{tablaObjetivos}{%
  \begin{center}
  \footnotesize % <-- tamaño intermedio, más legible que \scriptsize
  \rowcolors{2}{gray!10}{white}
  \begin{tabular}{|J{6cm}|J{6cm}|C{1.8cm}|C{1cm}|}
  \hline
  \CabeceraTabla
}{%
  \end{tabular}
  \end{center}
}


\newcommand{\ObjEsp}[3]{%
  \subsubsection*{%
    \textcolor{UoDBlue}{\textbf{•  Objetivo específico #1:}} #2
  }%
  \vspace{-4mm} % <-- reduce el espacio entre el título y la tabla
  \begin{tablaObjetivos}
    #3
  \end{tablaObjetivos}
}
% =====================================
% OBJETIVO ESTRATÉGICO CON BORDE IZQUIERDO NARANJA (QUE PUEDE CORTAR PÁGINA)
% =====================================


\newmdenv[
  linecolor=Naranja,         % color del borde
  linewidth=2pt,             % grosor de la línea
  topline=false,             % sin línea arriba
  bottomline=false,          % sin línea abajo
  rightline=false,           % sin línea derecha
  leftline=true,             % línea solo izquierda
  innertopmargin=5pt,        % espacio arriba
  innerbottommargin=5pt,     % espacio abajo
  innerleftmargin=8pt,       % espacio entre línea y texto
  innerrightmargin=5pt,      % margen derecho interno
  skipabove=10pt,            % espacio antes del bloque
  skipbelow=10pt,            % espacio después del bloque
  splittopskip=10pt,         % controla salto de página
  splitbottomskip=10pt,      % controla salto de página
  backgroundcolor=white,     % fondo blanco
  frametitlebackgroundcolor=white
]{objetivoBox}

% ---- Macro unificada para objetivo estratégico ----
% \newcommand{\ObjetivoEstrategico}[4]{%
%   \begin{objetivoBox}
%     % Espacio antes del título
%     \vspace{2mm} % <-- ajusta el valor según prefieras
%
%     % Título + descripción
%     \begin{minipage}[t]{0.75\textwidth}
%       {\color{Naranja}\textbf{Objetivo Estratégico #1:}}\\[1mm]
%       {\color{Naranja}\normalfont #2}
%     \end{minipage}%
%     \hfill
%     \begin{minipage}[t]{0.22\textwidth}
%       \vspace{0pt}
%       \begin{center}
%         \graficoCumplimientoChico{#3}
%       \end{center}
%     \end{minipage}
%
%     % Espacio antes de los objetivos específicos
%     \vspace{-6mm}
%
%     % Procesar lista de objetivos específicos
%     \forcsvlist{\ObjEspProcess}{#4}%
%   \end{objetivoBox}
% }

\newcommand{\ObjetivoEstrategico}[4]{%
  \begin{objetivoBox}
    \vspace{2mm}

    \ifstrempty{#3}{
      {
        \color{Naranja}\textbf{Objetivo Estratégico #1:}\\[1mm]
        \color{Naranja}\normalfont #2
      }
    }{%
      \begin{minipage}[t]{0.75\textwidth}
        {\color{Naranja}\textbf{Objetivo Estratégico #1:}}\\[1mm]
        {\color{Naranja}\normalfont #2}
      \end{minipage}%
      \hfill
      \begin{minipage}[t]{0.22\textwidth}
        \vspace{0pt}
        \begin{center}
          \graficoCumplimientoChico{#3}%
        \end{center}
      \end{minipage}
    }%

    \vspace{-6mm}

    \forcsvlist{\ObjEspProcess}{#4}%
  \end{objetivoBox}
}

% ============================
% GRÁFICOS DE CUMPLIMIENTO
% ============================
% Gráfico cumplimiento objetivo estratégico
% \newcommand{\graficoCumplimientoChico}[1]{%
% \begin{tikzpicture}
%   % Fondo gris
%   \draw[gray!20, line width=6pt] (0,0) circle (1.2cm);
%   % Calcular ángulo
%   \pgfmathsetmacro{\angulo}{#1*360}
%   % Progreso
%   \draw[Naranja, line width=6pt] (90:1.2cm) arc (90:{90-\angulo}:1.2cm);
%   % Texto dentro
%   \node at (0,0.35) {\scriptsize Cumplimiento};
% %   \node at (0,-0.1) {\textbf{#1\%}};
%   \node at (0,-0.1) {\SI{\fpeval{#1*100}}{\percent}};
% \end{tikzpicture}
% }
% Gráfico cumplimiento objetivo estratégico
% Gráfico cumplimiento objetivo estratégico
\newcommand{\graficoCumplimientoChico}[1]{%
\begin{tikzpicture}
  % Fondo gris
  \draw[gray!20, line width=6pt] (0,0) circle (1.2cm);

  % Calcular ángulo
  \pgfmathsetmacro{\angulo}{#1*360}

  % Determinar categoría numérica
  \pgfmathsetmacro{\catNum}{ifthenelse(#1<0.6,0,ifthenelse(#1<0.8,1,2))}

  % Asignar texto según categoría
  \ifnum\catNum=0 \def\categoria{BAJO} \def\colorCategoria{red!90!black}\fi
  \ifnum\catNum=1 \def\categoria{MEDIO} \def\colorCategoria{blue!80!black}\fi
  \ifnum\catNum=2 \def\categoria{NORMAL} \def\colorCategoria{green!50!black}\fi

  % Progreso con color
  \draw[\colorCategoria, line width=6pt] (90:1.2cm) arc (90:{90-\angulo}:1.2cm);

  % Texto dentro
  \node at (0,0.35) {\scriptsize Cumplimiento};
  \node at (0,-0.1) {\SI{\fpeval{#1*100}}{\percent}};

  % Mostrar categoría debajo
  \node at (0,-0.55) {\footnotesize \textbf{\categoria}};
\end{tikzpicture}
}

% Gráfico sección cumplimiento general
\newcommand{\graficoCumplimientoGeneral}[2]{%
\begin{tikzpicture}
  % Fondo gris
  \draw[gray!20, line width=7.5pt] (0,0) circle (1.4cm); % radio intermedio
  % Calcular ángulo
  \pgfmathsetmacro{\angulo}{(#2/100)*360}
  % Progreso
  \draw[Naranja, line width=7.5pt] (90:1.4cm) arc (90:{90-\angulo}:1.4cm);
  % Texto dentro
  \node at (0,0.35) {\footnotesize OE #1};     % número variable
  \node at (0,-0.2) {\Large\textbf{#2\%}};     % porcentaje más grande
\end{tikzpicture}%
}

%=========================
% CONCLUSIÓN OBJETIVO
%=========================
\newcommand{\ConclusionObj}[1]{%
  \vspace{-7.5mm} % quita espacio extra antes de la conclusión
  \begin{tcolorbox}[colback=blue!7, colframe=white,
                    title=Conclusión, fonttitle=\bfseries,
                    boxrule=0pt,
                    sharp corners,
                    width=\textwidth]
    {\color{black!70} #1} % <-- aquí el texto más suave
  \end{tcolorbox}%
  \vspace{2mm} 
}



%===========================
% TABLA FINAL, COMPARATIVA
%===========================

\newcommand{\FilaOE}[3]{#1 & #2 & #3 \\ \hline}

% Tabla de Objetivos Estratégicos: #1 = Año 1, #2 = Año 2, #3 = filas
\newcommand{\TablaObjetivosEstrategicos}[3]{%
\begin{center}
\footnotesize
\rowcolors{2}{gray!10}{white}%
\begin{tabular}{|J{12cm}|C{1.5cm}|C{1.5cm}|}
\hline
\rowcolor{UoDBlue}
\textcolor{white}{\textbf{Objetivos Estratégicos}} &
\textcolor{white}{\textbf{#1}} &
\textcolor{white}{\textbf{#2}} \\
\hline
#3
\end{tabular}
\end{center}%
}
